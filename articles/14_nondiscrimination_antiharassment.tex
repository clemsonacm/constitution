\section*{Article XIV: Non-Discrimination and Anti-Harassment Policy}
\addcontentsline{toc}{section}{Article XIV: Non-Discrimination and Anti-Harassment Policy}

\subsection*{Purpose}

This policy of Non-Discrimination is intended to meet Clemson University’s responsibilities under Titles VI and VII of the Civil Rights Act 1964, the Pregnancy Discrimination Act of 1978, Title IX of the Education Amendments of 1972, Sections 503 and 504 of the Rehabilitation Act of 1973, the Americans with Disabilities Act of 1990, the Age Discrimination in Employment Act of 1967, the Age Discrimination Act of 1975, the Vietnam Veterans Readjustment Assistance Act of 1974, the Genetic Information Nondiscrimination Act of 2008, the Violence Against Women Act, the SC Pregnancy Accommodations Act and applicable provisions of the South Carolina Human Affairs Law.

\subsection*{Scope}

This policy applies to all University students, employees, affiliates, and visitors.

\subsection*{Policy Statement}

\begin{enumerate}

	\item Clemson prohibits discrimination, including harassment, of any employee, student, guest or visitor because of race, color, religion, sex, sexual orientation, gender, gender identity, pregnancy (including childbirth, or related medical condition), national origin, age, disability, veteran’s status, genetic information, or any other personal characteristic protected under applicable federal or state law.

	\item Clemson University will respond promptly to all complaints of discrimination, harassment, or retaliation.

	\begin{enumerate}

		\item Any person, regardless of position or title, who is determined to have engaged in discrimination, harassment or retaliation as prohibited by this policy will be subject to prompt and appropriate corrective action, up to and including dismissal or termination from the University, or in the case of visitors, exclusion from University property and/or programs.

	\end{enumerate}

	\item Clemson also prohibits retaliation against any person because the person filed a complaint of discrimination or because the person participated in any manner in the investigation and resolution of a complaint of discrimination or harassment.

\end{enumerate}

\subsection*{Definitions}

\begin{enumerate}

	\item \textbf{Discrimination}: treating a person or group of persons less advantageously than another person or group of persons because of one or more of the protected characteristics listed above. Discrimination can manifest itself in many forms, including denying or excluding a person or a group of persons from participation in or receiving the benefits of any program or activity of the University, including employment decisions, because of one or more of the protected characteristics specified above.

	\item \textbf{Harassment}: is unwelcome verbal or physical conduct directed toward a person or group of persons motivated by a protected characteristic that is so severe, pervasive, or persistent, and objectively offensive that it unreasonably interferes with the person’s educational performance, or in an employment setting, that it unreasonably interferes with the person’s work performance or creates an intimidating or hostile work environment. Possible examples may include, but are not limited to, the following when they are part of a pattern of conduct that rises to the level of the standard set forth above: epithets, slurs, and jokes. Sexual harassment has its own definition (see below).

	\item \textbf{Sexual harassment}: is a particular type of harassment including unwelcome verbal or physical conduct of a sexual nature and as further defined in the Policy and Procedures Related to Sexual Harassment and Sexual Violence. Possible examples may include, but are not limited to, unwelcome verbal or physical conduct of a sexual nature, sexual advances, requests for sexual favors, touching, jokes, comments, and sexual violence when such conduct constitutes prohibited conduct as defined in the Policy and Procedures Related to Sexual Harassment and Sexual Violence.

	\item Speech or conduct alone protected under state or federal law will not be the basis for disciplinary action.
\end{enumerate}

\subsection*{Additional Resources}

\begin{enumerate}

	\item \href{https://clemsonpub.cfmnetwork.com/B.aspx?BookId=10891&PageId=453047}{Reporting and Inquiry Contact Information}

	\item \href{https://media.clemson.edu/humanres/policies_procedures/policy-and-procedures-related-to-sexual-harassment-and-sexual-violence-VAWA.pdf}{Policy and Procedures Related to Sexual Harassment and Sexual Violence}

	\item \hrefhttps://clemsonpub.cfmnetwork.com/B.aspx?BookId=12732&PageId=462857}{Procedures for Resolution of Discrimination/Harassment/Retaliation Complaints Against Employees (PDF)}
	
	\item \href{https://www.clemson.edu/studentaffairs/community-resources/oces/_documents/student_code_of_conduct.pdf}{Student Code of Conduct}

	\item \href{https://forms.office.com/pages/responsepage.aspx?id=9vibDK3Mh0uBjUkCaTiql351epYjGHtIrR1GOAFwwYhUMU9OSkVMQzhaMTQ2UFUyUzRIMUE0TU1XSi4u}{Discrimination, Harassment, and/or Retaliation Incident Reporting Form}

\end{enumerate}

\subsection*{Responsible Department}

Access Compliance and Education (5404), 864-656-3181

\subsection*{Approval \& Revision History}

\begin{enumerate}
	
	\item President Approval: 08/15/2022

	\item Last Date of Revision: 06/04/2024

	\item Originally Issued: 12/17/2018

\end{enumerate}
